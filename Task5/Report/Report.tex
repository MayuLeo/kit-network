\documentclass[11pt]{jsarticle}
\usepackage{fancyhdr}
\usepackage{amsmath,amssymb}
\pagestyle{fancy}
\lhead{}
\rhead{}
\rhead{\thepage{}}
\cfoot{}
\renewcommand{\headrulewidth}{0pt}
\usepackage{Mayu}

\begin{document}
    \section{はじめに}
        この設計書ではネットワークプログラミングの課題5である井戸端会議システムについての大まかな流れや各ソースファイルの概要の説明を記す.
    \section{プログラムの流れ}
        以下にこのプログラムの大まかな流れの図を示す.
        \pict{16}{flow.pdf}{プログラムの流れ}
\newpage
    \section{ソースファイルの概要}
        このプログラムは以下のソースファイルで構成されている.
        \begin{itemize}
            \item Conf.c
            \item Conf\_server.c
            \item Conf\_client.c
            \item Conf\_util.c
            \item Conf\_common.c
        \end{itemize}
        それぞれの説明を以下に示す.
    \subsection{Conf.c}
        これは初回起動時に実行されるmain文を含む.実行された後オプション文字列を取得しusernameを得る.その後,HELOパケットを送信し返事が帰って来なければConf\_server.cへ,返事が帰ってきたらConf\_client.cへ遷移する.
    \subsection{Conf\_server.c}
        自身がサーバーとなるときこれが呼ばれる.まずConf\_server()関数へ移動しスレッドを生成する.生成された先ではUDP通信を用いてHEROパケットを受信したらHEREパケットを送り返す.
        
        サーバー本体ではTCP通信により受け取ったパケットごとにヘッダを解析,それに対応した処理を行う.解析処理はMoodleに記載の課題5のヒントに従う.
        
    \subsection{Conf\_client.c}
        自身がクライアントとなるときこれが呼ばれる.Conf\_client()ではまず,サーバーにJOIN usernameメッセージを送信を送信する.その後,MESGが届けばそれを表示し,キーボードからの入力があればPOSTを付け足し送信する.この時,キーボード入力がQUITの時はそのままサーバーに送信し,Conf\_client.cは接続を終了したのちプログラムを終了する.
    
    \subsection{Conf\_util.c}
        このプログラムは各クライアントのユーザー情報を格納する構造体の定義と宣言を行う.宣言はグローバル変数として行う.ユーザーの構造体はユーザー名・ソケット番号・次のユーザーの3つの情報を持つ.次のユーザーが存在しない場合はNULLを指定する.
        
    \subsection{Conf\_common.c}
        このプログラムはacceptやsendといった関数をエラー処理と一緒に呼び出せるようまとめた関数を揃える.必要に応じて各プログラムから呼び出す.
\end{document}
